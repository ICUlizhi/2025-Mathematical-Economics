\begin{comment}
    

\documentclass[lang=cn,10pt,green]{elegantbook} 
\title{2025年数理经济学笔记}
\subtitle{授课: 杨佳楠老师}

\author{徐靖}
\institute{PKU}
\date{Febuary 27, 2025}
\bioinfo{声明}{请勿用于个人学习外其他用途!}

\extrainfo{个人笔记, 如有谬误, 欢迎指正! 联系方式 : 2200012917@stu.pku.edu.cn}

\setcounter{tocdepth}{3}

\logo{logo-blue.png}
\cover{cover.jpg}

% 本文档命令
\usepackage{array}
\newcommand{\ccr}[1]{\makecell{{\color{#1}\rule{1cm}{1cm}}}}

% 修改标题页的橙色带
% \definecolor{customcolor}{RGB}{32,178,170}
% \colorlet{coverlinecolor}{customcolor}

\begin{document}

\maketitle
\frontmatter

\tableofcontents

\mainmatter
\end{comment}
% TODO

\chapter{Multi-Variable Calculus\footnote{多元微分还能有不会的吗, 这真不用记了吧}}

\begin{introduction}[Keywords]
    \item gradient 梯度
\end{introduction}
\section{introduction}
\subsection{Motivation and Insight}
\begin{itemize}
    \item Many practical problems involve optimization with multple variables.
    \item Real-world applications often require optimizing several variables simultaneoutly.
    \item Linear functions are easy to understand and manipulate.
    \begin{itemize}
        \item Not all interesting functions are linear, but many can be approximated by linear functions.
        \item The gradient is a generalization of the derivative to functions of multiple variables.
    \end{itemize}
\end{itemize}



%\end{document}







