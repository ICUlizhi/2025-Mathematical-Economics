\begin{comment}
    

\documentclass[lang=cn,10pt,green]{elegantbook} 
\title{2025年数理经济学笔记}
\subtitle{授课: 杨佳楠老师}

\author{徐靖}
\institute{PKU}
\date{Febuary 27, 2025}
\bioinfo{声明}{请勿用于个人学习外其他用途!}

\extrainfo{个人笔记, 如有谬误, 欢迎指正! 联系方式 : 2200012917@stu.pku.edu.cn}

\setcounter{tocdepth}{3}

\logo{pkuhub-cn.png}
\cover{cover.jpg}


% 本文档命令
\usepackage{array}
\newcommand{\ccr}[1]{\makecell{{\color{#1}\rule{1cm}{1cm}}}}

% 修改标题页的橙色带
% \definecolor{customcolor}{RGB}{32,178,170}
% \colorlet{coverlinecolor}{customcolor}

\begin{document}

\maketitle
\frontmatter

\tableofcontents

\mainmatter
% \end{comment}
% TODO

\chapter{Multi-Variable Optimization with Equality Constraints}

\begin{introduction}[Keywords]
    \item Equality Constraints 等式约束
    \item Lagrange Multiplier 拉格朗日乘数法
    \item Nondegenerate Constraint Qualification, NDCQ 非退化约束条件
    \item Cobb-Douglas Utility Function 柯布-道格拉斯效用函数
\end{introduction}
\begin{note}
现在我们考虑有约束条件的优化问题, 这一关键是将约束视为函数方程并引入拉格朗日乘数法. 
\end{note}
\begin{definition}[Optimization with Equality Constraints]
    设 $f(x)$ 是可微函数组, $g(x) = 0$ 是可微约束条件组, 那么我们要优化的问题可以表示为:
    \begin{equation}
        \max f(x) \quad s.t. \quad g(x) = 0
    \end{equation}
    设 f 是 $n$ 维向量, $g$ 是 $m$ 维向量, x 是 $k$ 维向量. 
\end{definition}
\begin{definition}[Lagrange Multiplier]
    对于上述问题, 我们可以构造拉格朗日函数:
    \begin{equation}
        L(x, \lambda) = f(x) + \lambda^T g(x)
    \end{equation}
    其中 $\lambda$ 是拉格朗日乘数.
    通过对 $L$ 求导数, 我们可以得到一组方程:
    \begin{equation}
        \frac{\partial L}{\partial x} = 0, \quad \frac{\partial L}{\partial \lambda} = g(x) = 0
    \end{equation}
    其解 $(x^*, \lambda^*)$ 就是我们要找的最优解.
\end{definition}
\begin{definition}[NDCQ]
    如果 $g(x)$ 在 $x^*$ 处可微, 且 $Dg(x^*)$ 的秩为 $m$, 那么我们称 $g(x)$ 满足\textbf{非退化约束条件 (NDCQ)}. 
    其中, $$Dg(x^*) = \begin{pmatrix}
        \frac{\partial g_1}{\partial x_1} & \cdots & \frac{\partial g_1}{\partial x_k} \\
        \vdots & \ddots & \vdots \\
        \frac{\partial g_m}{\partial x_1} & \cdots & \frac{\partial g_m}{\partial x_k}
    \end{pmatrix}$$
\end{definition}
\begin{theorem}[Lagrange Multiplier Theorem]
    设 $f(x)$ 和 $g(x)$ 都是可微函数, 且 $g(x)$ 满足非退化约束条件. 那么 $(x^*, \lambda^*)$ 是上述优化问题的最优解.
\end{theorem}
\begin{note}
    我们需要进一步判断最大值还是最小值.
\end{note}
\begin{definition}[Borderde Hessian Matrix]
    \begin{equation}
        H = \begin{pmatrix}
            0 & Dg(x^*) \\
            Dg(x^*)^T & D^2 f(x^*)
        \end{pmatrix}
    \end{equation}
    其中 $D^2 f(x^*)$ 是 $f(x)$ 在 $x^*$ 处的 Hessian 矩阵, $Dg(x^*)$ 是 $g(x)$ 在 $x^*$ 处的 Jacobian 矩阵.

    本质是求 Lagrange 函数的 Hessian 矩阵, 它是 $k+m$ 维的.
\end{definition}
\begin{theorem}[Sufficient Condition for Maximum]
    设 $H$ 是上述的 Borderde Hessian 矩阵, 那么如果 $H$ 是正定的, 那么 $(x^*, \lambda^*)$ 是最大值.
    如果 $H$ 是负定的, 那么 $(x^*, \lambda^*)$ 是最小值.
\end{theorem}

%\end{document}







