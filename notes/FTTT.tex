\begin{comment}
    

\documentclass[lang=cn,10pt,green]{elegantbook} 
\title{2025年数理经济学笔记}
\subtitle{授课: 杨佳楠老师}

\author{徐靖}
\institute{PKU}
\date{Febuary 27, 2025}
\bioinfo{声明}{请勿用于个人学习外其他用途!}

\extrainfo{个人笔记, 如有谬误, 欢迎指正! 联系方式 : 2200012917@stu.pku.edu.cn}

\setcounter{tocdepth}{3}

\logo{pkuhub-cn.png}
\cover{cover.jpg}


% 本文档命令
\usepackage{array}
\newcommand{\ccr}[1]{\makecell{{\color{#1}\rule{1cm}{1cm}}}}

% 修改标题页的橙色带
% \definecolor{customcolor}{RGB}{32,178,170}
% \colorlet{coverlinecolor}{customcolor}

\begin{document}

\maketitle
\frontmatter

\tableofcontents

\mainmatter
% \end{comment}
% TODO


\chapter{From Thick to Thin}
笔者的一些insights, 便于理解和记忆数理经济学的体系, 也是考前的复习提纲
\section*{期中部分}

\colorbox{yellow}{前三章} 介绍了前置数学工具, 包括线性代数, 欧式空间(的拓扑性质), 以及多元微分. 其中相对核心的有:

\begin{itemize}
    \item 正定矩阵及其判定, 对角化的一系列结论
    \item Bolzano-Weierstrass Theorem
    \item Hessian, Jacobian, lagrange
\end{itemize}

\subsection*{1. 极值与零点}
\begin{itemize}
    \item \textbf{无约束优化}: $f: \mathbb{R}^n \to \mathbb{R}$,求$x^*$使$f(x^*)$最小
    \item \textbf{函数组零点}: $F: \mathbb{R}^n \to \mathbb{R}^n$,求$x^*$使$F(x^*)=0$
\end{itemize}

以上两个问题的交集在于 FOC: 
$$ F = \nabla f = 0 $$

\begin{note}
换句话说, 优化问题求解的第一步是零点问题, 零点问题的一些特例可以还原成优化问题 (显然不是所有的方程组都是某个函数的 FOC), 假如 $F = \nabla f$, 有:
\begin{itemize}
    \item $f$ 的 Hessian 矩阵是 $F$ 的Jacobian 矩阵, 他们都是对称矩阵
    $$
    \begin{pmatrix}
        \frac{\partial^2 f}{\partial x_1^2} & \cdots & \frac{\partial^2 f}{\partial x_1 \partial x_n} \\
        \vdots & \ddots & \vdots \\
        \frac{\partial^2 f}{\partial x_n \partial x_1} & \cdots & \frac{\partial^2 f}{\partial x_n^2}
    \end{pmatrix} = \begin{pmatrix}
        \frac{\partial F_1}{\partial x_1} & \cdots & \frac{\partial F_1}{\partial x_n} \\
        \vdots & \ddots & \vdots \\
        \frac{\partial F_n}{\partial x_1} & \cdots & \frac{\partial F_n}{\partial x_n}
    \end{pmatrix}
$$
\end{itemize}
\end{note}

\subsubsection*{1.1 零点求解方法}
\colorbox{yellow}{第四章} 介绍了五种方法
\begin{itemize}
    \item \textbf{Bisection}: 二分法, $x_{n+1} = \frac{x_n + x_{n-1}}{2}$
    \item \textbf{Secant}: 割线法, $x_{n+1} = x \text{axis} \cap \text{line}(x_n, x_{n-1})$
    \item \textbf{False Position}: 假位法, 是割线法的变种, 每次都在 $x_n$ 和 $x_{n-1}$ 之间取一个点与 $x_{n+1}$ 构成新区间, 保证区间含根
    \item \textbf{Newton Raphson}: 牛顿法, $x_{n+1} = x_n - \frac{f(x_n)}{f'(x_n)}$
    \item \textbf{Gradient Descent}: 梯度下降法, $x_{n+1} = x_n - t \nabla f(x_n)$, 其中 $t = \arg\min_{t > 0} f(x_n - t \nabla f(x_n))$
    
\end{itemize}

\subsubsection*{1.2 极值点判定}
规范方法 : 判断 Hessian 矩阵的正定性, 本课程用顺序主子式法\footnote{正小负大}来判断会比较快
$$
\begin{array}{|c|c|c|}
\hline
H_f & H_i & x^* \\ \hline
\text{正定} & H_i > 0 & \text{极小值} \\ \hline
\text{负定} & H_i < 0 & \text{极大值} \\ \hline
\text{不定} & H_i \neq 0, \text{但不属于以上两种} & \text{鞍点} \\ \hline
\end{array}
$$


实用技巧: 
\begin{itemize}
    \item 正定矩阵的特征值都是正数, 负定矩阵的特征值都是负数
    \item 先代入临界点排除明显非极值情况
    \item  结合 $f$ 的凹凸性判断, 如果严格凸或者严格凹, 则临界点一定是极小值或者极大值
\end{itemize}

\subsection*{2. 无约束优化与约束优化}

约束优化是\colorbox{yellow}{第五章}的内容, 即在可微函数\textbf{组} $g(x) = 0$ 上求 $f$ 的极值, 沟通二者的桥梁是拉格拉日乘数法与 \textbf{NDCQ}
$$ L(x, \lambda) = f(x) + \lambda^\top g(x) $$
然后 $(x^*, \lambda^*)$ 是一阶条件 $\nabla L(x^*, \lambda^*) = 0$ 的解, 我们似乎回到了无约束优化

\subsubsection*{2.1 非退化约束条件} 
NDCQ (非退化约束条件) 是让拉格朗日乘数法成立的条件
\begin{theorem}
假设 $x$ 是 $k$ 维向量, $\lambda$ 是 $m$ 维向量, 则 $L$ 是 $k + m$ 维向量, 考虑
$$
Dg(x^*) = 
\begin{pmatrix}
\frac{\partial g_1}{\partial x_1} & \cdots & \frac{\partial g_1}{\partial x_k} \\
\vdots & \ddots & \vdots \\
\frac{\partial g_m}{\partial x_1} & \cdots & \frac{\partial g_m}{\partial x_k}
\end{pmatrix}
$$
\textbf{NDCQ: } $\text{rank}(Dg(x^*)) = m$
\begin{itemize}
    \item 假如 \textbf{NDCQ} 得到满足, 则 $(x^*, \lambda^*)$ 可能是极值点
    \item 假如不满足, 即约束 $g$ 是退化的, 此时可能仍然存在极值点, 但无法由拉格朗日乘数法得到
\end{itemize}
\end{theorem}
  
\subsubsection*{2.2 加边海色矩阵} 
加边海色矩阵是判断有约束优化问题极值点的工具:
$$
\bar{H} = 
\begin{pmatrix}
0 & Dg(x^*) \\
Dg(x^*)^\top & H_f(x^*)
\end{pmatrix}
$$ 

加边后显然不正定了, 判断方法\footnote{判断方法和完整证明参考\href{https://sites.math.northwestern.edu/~clark/285/2006-07/handouts/lagrange-2deriv.pdf}{Northwestern University 的笔记}. 实际上, 课堂上只讲了 $m = 1$ 的情形, 在此情形下极小值对应主子式恒负, 极大值对应符号交替, 这样就不对称, 和无约束优化问题的海色矩阵对比起来显得不自然. 经济学中最简单的 $m=1, n=2$ 的情形, 自然只需看 $|\bar{H}|$ 的符号, 正大负小}也有变化 :
$$
\begin{array}{|c|c|c|}
\hline
\bar{H} & H_{i}, i\in [2m+1,n+m] & x^* \\ \hline
 & \text{sgn} H_{i} = (-1)^{m} & \text{极小值} \\ \hline
 & \text{sgn} H_{i} = (-1)^{i-m} & \text{极大值} \\ \hline
 & \text{sgn} H_{i} \neq 0, \text{但不属于以上两种} & \text{鞍点} \\ \hline
\end{array}
$$
\begin{note}
对加边海色矩阵的一个直观理解\footnote{这个直观理解来自于 \href{https://www.math.cmu.edu/~hanifc/NotesOnHessians.pdf}{CMU 的笔记}} : 假如 $f$ 在 $g = 0$ 这个$n-m$ 维的嵌入 $\mathbb{R}^n$ 的流形上表现出了类似正定负定的性质, 那么可以判断是极值还是鞍点, 比如我们在 3 维空间中的球面上找极值.
\end{note}

做题时如果从几何解释出发, 可以避免犯错

\subsection*{3. 外生变量}
\colorbox{yellow}{第六章} 的内容探究外生变量的作用. 一是探究外生变量 (作为经济模型的输入) 对内生变量 (经济模型的输出) 的影响 (偏导数), 二是外生变量对最优值函数 $V=\max f$ 的影响 (包络定理).

打个比方, 企业的生产存在外部因素(劳动力价格, 资本价格等), 决策得到的产量, 最终的成本/收益. 这三者分别对应外生变量, 内生变量, 最优值函数.

\subsection*{3.1 对内生变量的影响}
无论无约束还是有约束, 对于多元函数 $f$ (或 $L = f+\lambda g$), 和方程组 $ F = \nabla f = 0$, 对外生变量 $\alpha$ 求导有:
$$\sum_{i=1}^{n} \frac{\partial F_k}{\partial x_i}\frac{\partial x_i}{\partial \alpha} + \frac{\partial F_k}{\partial \alpha}=0$$

整理得:
$$J \frac{\partial x^*}{\partial \alpha} + \frac{\partial F}{\partial \alpha} =0$$

其中 $J$ 是 $f$ 的 Hessian 矩阵 (Bordered Hessian 矩阵), 也是 $F$ 的 Jacobian 矩阵. 规定 $J_i$ 是用 $F$ 取代原本第 $i$ 列后的矩阵. 利用隐函数定理和克拉默法则, 我们可以得到:
$$ \frac{\partial x^*_i}{\partial \alpha} = -\frac{\det J_{i}}{\det J} $$

\subsection*{3.1 对最优值函数的影响}
外生变量对最优值函数的影响, 包含直接影响和通过内生变量作用的间接影响两部分. 而包络定理揭示了后者在最优解处趋于零 (因为 $\nabla f = 0$), 因此只需考虑前者. 
$$\frac{\partial V(a)}{\partial a} = \underbrace{\frac{\partial f}{\partial a}}_{\text{直接影响}} + \underbrace{\nabla_x f \cdot \frac{\partial x^*(a)}{\partial a}}_{\text{间接影响}} $$

形象的经济学理解比如 Shephard's Lemma : 当要素的价格上涨 1 单位时, 最小成本的边际增加量正好等于厂商对该要素的使用量.
\section*{期末部分}



%\end{document}







