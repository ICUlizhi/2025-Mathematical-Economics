\begin{comment}
    

\documentclass[lang=cn,10pt,green]{elegantbook} 
\title{2025年数理经济学笔记}
\subtitle{授课: 杨佳楠老师}

\author{徐靖}
\institute{PKU}
\date{Febuary 27, 2025}
\bioinfo{声明}{请勿用于个人学习外其他用途!}

\extrainfo{个人笔记, 如有谬误, 欢迎指正! 联系方式 : 2200012917@stu.pku.edu.cn}

\setcounter{tocdepth}{3}

\logo{logo-blue.png}
\cover{cover.jpg}

% 本文档命令
\usepackage{array}
\newcommand{\ccr}[1]{\makecell{{\color{#1}\rule{1cm}{1cm}}}}

% 修改标题页的橙色带
% \definecolor{customcolor}{RGB}{32,178,170}
% \colorlet{coverlinecolor}{customcolor}

\begin{document}

\maketitle
\frontmatter

\tableofcontents

\mainmatter
\end{comment}
% TODO
\chapter{Topology of $\mathbb{R}^N$\footnote{点集拓扑对应数分高代这一级别的数学基础课, 只有sms和图班的同学学过, 所以我简单记一下, 无需关注证明, 数理经济学只用到结论}}

\begin{introduction}[Keywords]
    \item Topology 拓扑
    \item Metric Space 度量空间
    \item Convergence 收敛
    \item interior 内部
    \item closure 闭包
    \item boundary 边界
    \item compact set 紧集
    \item cluster point 聚点
    \item Lipschitz continuity 利普希茨连续
    \item semicontinuity 半连续
    \item Bolzano-Weierstrass Theorem 博尔扎诺-魏尔斯特拉斯定理
    \item Heine-Borel Theorem 海涅-波雷尔定理
    \item Contraction Mapping Theorem 压缩映射定理
    \item Intermediate Value Theorem 中值定理
\end{introduction}

\section{Metric Spaces}
\subsection{Definition of Metric Spaces}
\begin{definition}
    Let $X$ be a set. A function $d: X \times X \rightarrow \mathbb{R}$ is called a \textbf{metric} (or \textbf{distance}) on $X$ if :
    \begin{enumerate}
        \item (positivity) $d(x,y) \geq 0$ for all $x,y \in X$ and $d(x,y) = 0$ if and only if $x = y$.
        \item (symmetry) $d(x,y) = d(y,x)$ for all $x,y \in X$.
        \item (triangle inequality) $d(x,y) \leq d(x,z) + d(z,y)$ for all $x,y,z \in X$.
    \end{enumerate}
\end{definition}
A set $X$ together with a metric $d$ is called a \textbf{metric space}, denoted by $(X,d)$.

\subsection{Examples of metrics in $\mathbb{R}^N$}
\begin{itemize}
    \item \textbf{ Euclidean metric}: $d(x,y) = \sqrt{\sum_{i=1}^{N} (x_i - y_i)^2}$.
    \item $L^p$ \textbf{metric} (for $p\ge 1$): $d(x,y) = (\sum_{i=1}^{N} |x_i - y_i|^p)^{1/p}$.
    \item \textbf{Sup norm} (when $p=\infty$): $d(x,y) = \max_{i=1}^{N} |x_i - y_i|$.
\end{itemize}

\section{Convergence of sequences}
\subsection{Definition of Convergence}
\begin{definition}
    Let $(X,d)$ be a metric space. A sequence $\{x_n\}$ in $X$ is said to \textbf{converge} to a point $x \in X$ if for every $\epsilon > 0$, there exists an integer $N$ such that $d(x_n,x) < \epsilon$ for all $n \geq N$. In this case, we write $\lim_{n \rightarrow \infty} x_n = x$. A sequence that converges is called \textbf{convergent}, otherwise it is called \textbf{divergent}.
\end{definition}

\begin{definition}
    When metric space is $\mathbb{R}^N$, we say that $\{x_n\}$ is \textbf{bounded} if there exists a real number $M$ such that $\|x_k\| \leq M$ for all $n$.
\end{definition}
\subsection{Cauchy Sequences and Complete Metric Spaces}
\begin{itemize}
    \item \textbf{Cauchy sequence}: A sequence $\{x_n\}$ in a metric space $(X,d)$ is called a \textbf{Cauchy sequence} if for every $\epsilon > 0$, there exists an integer $N$ such that $d(x_n,x_m) < \epsilon$ for all $n,m \geq N$.
    \item \textbf{Complete metric space}: A metric space $(X,d)$ is called \textbf{complete} if every Cauchy sequence in $X$ converges to a point in $X$.
\end{itemize}
\begin{theorem}
    Any convergent sequence in a metric space is a Cauchy sequence.
\end{theorem}
\subsection{Example: Cauchy Sequence Not Convergent in $\mathbb{Q}$}
Consider the metric space $(\mathbb{Q},d)$, where $d(x,y) = |x-y|$. 

\textbf{Fibonacci sequence} :
Let $\{F_k\}$ be the Fibonacci sequence, defined by $$F_1 = F_2 = 1, F_{k+1} = F_k + F_{k-1}, k \geq 2$$ 

\textbf{A Special Sequence}:
Define $a_k = \frac{F_{k+1}}{F_k}$. Then $\{a_k\}$ is a Cauchy sequence in $\mathbb{Q}$ but does not converge in $\mathbb{Q}$.
\subsection{Properties of Convergent Sequences in $\mathbb{R}^N$}
Consider $\mathbb{R}^N$ with the Euclidean metric. Let $\{x_n\}$ and $\{y_n\}$ be two sequences.
\begin{itemize}
    \item \textbf{Preservation of Addition/Subtraction}: If $\lim_{n \rightarrow \infty} x_n = x$ and $\lim_{n \rightarrow \infty} y_n = y$, then $\lim_{n \rightarrow \infty} (x_n \pm y_n) = x \pm y$.
    \item \textbf{Preservation of Multiplication}: If $\lim_{n \rightarrow \infty} x_n = x$ and $\lim_{n \rightarrow \infty} y_n = y$, then $\lim_{n \rightarrow \infty} (x_n \cdot y_n) = x \cdot y$.
    \item \textbf{Preservation of Division}: If $\lim_{n \rightarrow \infty} x_n = x$ and $\lim_{n \rightarrow \infty} y_n = y \neq 0$, then $\lim_{n \rightarrow \infty} \frac{x_n}{y_n} = \frac{x}{y}$.
    \item \textbf{Preservation of Inequality}: If $\lim_{n \rightarrow \infty} x_n = x$ and $\lim_{n \rightarrow \infty} y_n = y$, then $x_n \leq y_n$ for all $n$ implies $x \leq y$.
\end{itemize}

\subsection{Properties of Sequences in $\mathbb{R}^N$}
\begin{property}
A convergent sequence in $\mathbb{R}^N$ is bounded.
\end{property}

A sequence $\{x_{n_k}\}$ is called a \textbf{subsequence} of $\{x_n\}$ if $n_1 < n_2 < n_3 < \cdots$.

\begin{property}
    subsequences of a convergent sequence in $\mathbb{R}^N$ also converge to the same limit.
\end{property}

\subsection{Limit Superior and Limit Inferior}
\begin{definition}
    Let $\{x_n\}$ be a sequence in $\mathbb{R}^N$. The \textbf{limit superior} of $\{x_n\}$ is defined by $$\limsup_{n \rightarrow \infty} x_n = \lim_{n \rightarrow \infty} \left( \sup_{k \geq n} x_k \right)$$
    The \textbf{limit inferior} of $\{x_n\}$ is defined by $$\liminf_{n \rightarrow \infty} x_n = \lim_{n \rightarrow \infty} \left( \inf_{k \geq n} x_k \right)$$
\end{definition}

\section{Topological properties}
\subsection{Open and Closed Sets}
\begin{definition}
    In a metric space $(X,d)$, a set $U \subset X$ is called \textbf{open} if for every $x \in U$, there exists an $\epsilon > 0$ such that $B(x,\epsilon) \subset U$. A set $F \subset X$ is called \textbf{closed} if its complement $F^c \overset{\text{def}}{=} X \backslash F$ is open.
\end{definition}
\begin{property}
    For open sets:
    \begin{enumerate}
        \item The union of any collection of open sets is open.
        \item The intersection of finitely many open sets is open.
    \end{enumerate}
    For closed sets:
    \begin{enumerate}
        \item The intersection of any collection of closed sets is closed.
        \item The union of finitely many closed sets is closed.
    \end{enumerate}
\end{property}
\subsection{Interior, Closure, and Boundary of Sets}
\begin{definition}
    The \textbf{interior} of a set $A \subset X$ is defined as: $$\text{int}(A) = \bigcup \{U \subset A: U \text{ is open}\}$$
    The \textbf{closure} of a set $A \subset X$ is defined as: $$\overline{A} = \bigcap \{F \supset A: F \text{ is closed}\}$$
    The \textbf{boundary} of a set $A \subset X$ is defined as: $$\partial A = \overline{A} \backslash \text{int}(A)$$
\end{definition}

\begin{proposition}
    \begin{itemize}
        \item $A \subset X$ is open if and only if $\partial A \subset A$.
        \item $A \subset X$ is closed if and only if $\partial A \subset A$.
    \end{itemize}
\end{proposition}
\subsection{Bounded Sets and Compact Sets in $\mathbb{R}^N$}
\begin{definition}
    A set $A \subset \mathbb{R}^N$ is called \textbf{bounded} if there exists a real number $M$ such that $\|x\| \leq M$ for all $x \in A$.

    A set $A \subset \mathbb{R}^N$ is called \textbf{compact} if for any sequence $\{x_n\}$ in $A$, there exists a subsequence $\{x_{n_k}\}$ that converges to a point in $A$.
\end{definition}

\begin{theorem} [Heine-Borel Theorem]
    In $\mathbb{R}^N$, a set $A$ is compact if and only if it is closed and bounded.
\end{theorem}

\section{Continuous functions}
\subsection{Cluster Points in Metric Spaces}
\begin{definition}
    Let $(X,d)$ be a metric space and $A \subset X$. A point $x \in X$ is called a \textbf{cluster point} of $A$ if for every $\epsilon > 0$, there exists a point $y \in A$ such that $d(x,y) < \epsilon$ and $x \neq y$.

    Equivalently, $x$ is a cluster point of $A$ if there exists a sequence $\{x_n\}$ in $A$ such that $\lim_{n \rightarrow \infty} x_n = x$ and $x_n \neq x$ for all $n$.
\end{definition}

\subsection{Limits of Functions at Cluster Points}
\begin{definition}
    Let $(X,d)$ and $(Y,\rho)$ be metric spaces, $A \subset X$, $f: A \rightarrow Y$, and $x$ be a cluster point of $A$. We say that $f$ has a \textbf{limit} $y \in Y$ at $x$ if for every $\epsilon > 0$, there exists a $\delta > 0$ such that $\rho(f(x_0),y) < \epsilon$ for all $x_0 \in A$ such that $0 < d(x_0,x) < \delta$.

    Equivalently, using neighborhoods: $f$ has a limit $y$ at $x$ if for every neighborhood $V$ of $y$, there exists a neighborhood $U$ of $x$ such that $f(U \cap A) \subset V$.
\end{definition}
\begin{property}
    \begin{enumerate}
        \item $\lim_{x\to \bar{x}} f(x) = f(\bar{x})$ if and only if for every sequence $\{x_n\}$ in $A$ such that $\lim_{n \to \infty} x_n = \bar{x}$, we have $\lim_{n \to \infty} f(x_n) = f(\bar{x})$.
        \item If $f$ has a limit at $x$, then the limit is unique.
    \end{enumerate}
\end{property}

\subsection{Continuity of Functions}
\begin{definition}
    Let $(X, d)$ and $(Y, \rho)$ be metric spaces, and $f : X \to Y$.
\begin{itemize}
    \item $f$ is \textbf{continuous at} $\bar{x} \in X$ if:
    $$
    \forall \varepsilon > 0, \exists \delta > 0 : \forall x \in X, d(x, \bar{x}) < \delta \Rightarrow \rho(f(x), f(\bar{x})) < \varepsilon
    $$
    Equivalently:
    $$
    \forall \varepsilon > 0, \exists \delta > 0 : f(B_\delta(\bar{x})) \subseteq B_\varepsilon(f(\bar{x}))
    $$
    \item $f$ is \textbf{continuous on} $X$ (or simply \textbf{continuous}) if:
    $$
    \forall \bar{x} \in X, f \text{ is continuous at } \bar{x}
    $$
\end{itemize}
\end{definition}

\begin{proposition}
    Let $(X,d)$ and $(Y,\rho)$ be metric spaces, $f : X \to Y$, and $x \in X$. The following are equivalent:
    \begin{enumerate}
        \item $f$ is continuous at $x$.
        \item For every sequence $\{x_n\}$ in $X$ such that $\lim_{n \to \infty} x_n = x$, we have $\lim_{n \to \infty} f(x_n) = f(x)$.
        \item For every open set $V \subset Y$, $f^{-1}(V)$ is open in $X$.
    \end{enumerate}
\end{proposition}
\subsection{Bolzano-Weierstrass Theorem}
\begin{theorem}[Bolzano-Weierstrass Theorem]
    If $K \subset \mathbb{R}^N$ is compact and nonempty, and $f : K \to \mathbb{R}^M$ is continuous, then :
    \begin{enumerate}
        \item $f(K)$ is compact.
        \item $f$ attains its maximum and minimum on $K$.
    \end{enumerate}
\end{theorem}

\subsection{Semicontinuity}
\begin{definition}
    For $f:\mathbb{R}^N \to \mathbb{R}^M$ :
    \begin{itemize}
        \item $f$ is \textbf{upper semicontinuous} at $x$ if :
        $$ f(x) \leq \limsup_{y \to x} f(y) \text{ for all } x \in \mathbb{R}^N$$
        \item $f$ is \textbf{lower semicontinuous} at $x$ if :
        $$ f(x) \geq \liminf_{y \to x} f(y) \text{ for all } x \in \mathbb{R}^N$$
    \end{itemize}
\end{definition}
\begin{remark} $f$ is upper semicontinuous $\Leftrightarrow -f$ is lower semicontinuous. \end{remark}
\begin{theorem}[Extrema of semicontinuous Functions]
    Let $K \subset \mathbb{R}^N$ be compact and $f: K \to \mathbb{R}$ be upper semicontinuous. Then $f$ attains its maximum on $K$. If $f$ is lower semicontinuous, then $f$ attains its minimum on $K$.
\end{theorem}

\subsection{Lipschitz Continuity}
\begin{definition}
    A function $f: \mathbb{R}^N \to \mathbb{R}^M$ is called \textbf{Lipschitz continuous} if there exists a constant $K > 0$ such that:
    $$\|f(x) - f(y)\| \leq K \|x - y\| \text{ for all } x,y \in \mathbb{R}^N$$
    where $K$ is called the \textbf{Lipschitz constant} of $f$. If $K < 1$, then $f$ is called a \textbf{contraction mapping}.
\end{definition}
\begin{remark}
    Lipschitz continuity implies uniform continuity, but the converse is not true. For example, $f(x) = x^2 $.
\end{remark}

\begin{theorem}[Contraction Mapping Theorem]
    Let $(X, d)$ be a complete metric space and $f: X \to X$ be a contraction mapping. Then $f$ has a unique fixed point $x^* \in X$, i.e., $f(x^*) = x^*$.
\end{theorem}


\begin{theorem}[Intermediate Value Theorem]
    Let $f: D \to \mathbb{R}$ be a continuous function and $D \subset \mathbb{R}$. If :
    \begin{itemize}
        \item $[a,b] \subset D$ (closed interval)
        \item $y$ is between $f(a)$ and $f(b)$
    \end{itemize}
    then there exists a point $c \in [a,b]$ such that $f(c) = y$.
\end{theorem}
%\end{document}








